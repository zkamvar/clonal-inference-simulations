\documentclass[letterpaper, draft, 10pt]{article}
\usepackage{lineno}
\usepackage[top = 3cm, bottom = 3cm, left = 2.5cm, right = 2.5cm]{geometry}


\title{Simulation of partially clonal diploid populations with simuPOP v.1.0.8}
\author{Zhian N. Kamvar}
\date{\today}


\begin{document}
\maketitle
\linenumbers

\section{Methods}
\subsection*{Introduction}

SimuPOP v.1.0.8 is a python scripted, forward-time, individual-based simulator that can simulation populations with any marker type. 
It works not by drawing data points from a distribution, but rather by initializing representatives of individuals and having these individuals undergo mating with user specified population parameters.
Often, it is used for traditional population genetics analyses such as pedigree or genetic drift simulations.
Simulations through simupop undergo a two step process, they are first initialized by user defined or random processes, then `evolved' through a specific mating and mutation regiment.
We simulated 100 replicates of populations with increasing levels of clonal reproduction over 10 loci with 6 to 10 alleles per locus and 10,000 individuals each. 
These populations were evolved under the different levels of clonal reproduction and then sub sampled 10 times with the sample sizes of 10, 25, 50, and 100 individuals. 

\subsection*{Population Intialization}
All of the clonal scenarios were simulated from the same 100 seed populations.


\end{document}
